\documentclass[../stegner_thesis.tex]{subfiles}

\begin{document}

\chapter{Conclusion}%
\label{ch:conclusion}

\par In this research, we have explored the definition of context in malware
detection and two proof-of-concept context-aware models.
These models use LDA to extract structural features from the code, combining
those features with contextual features to detect context violations.
The first model utilizes the system's physical context, while the second
compares the expected and actual behavior of the programs.
These models are successfully able to detect their respective context
violations.

\par The primary takeaway from this work is that it is difficult to define
precisely what context is in malware analysis.
In fact, a significant endeavor on this research was just determining how to
frame context for software.
We framed the idea of context as a series of questions which we should answer,
but they do not translate directly into a computational model.
The proof-of-concept models we presented showed strong performance at their
designated tasks, but they do not make up a complete picture of context,
leaving room for a multitude of future research directions.

\section{Future Work}%
\label{sec:concl_future}

\subsection{Practical Example}%
\label{sec:concl_practical_example}

\par The example scenarios presented in this work were at an abstract level and
do not extend well to practical examples.
To continue working on context-aware models, a more concrete, practical example
must be explored.
One of the biggest barriers in testing a practical example is the nature of
publicly available datasets.
These datasets are not designed with context in mind, so they do not include
the necessary information required to establish context.
Ideally, each file in the dataset should be labeled with contextual
information, such as the intended behavior of the file.
Such a dataset would prove invaluable to the development of future
context-aware models.

\subsection{Dynamic Analysis}%
\label{sec:concl_dynamic}

\par In this work, we only utilized static analysis techniques.
For future work, it would be advantageous to include dynamic analysis features
and use them in conjunction with the static analysis.
Combining the two types of features could improve the robustness of the models.
Our initial model included dynamic assembly trace features and high-level
behavioral features, but these features were not included in the final model
due to time limitations in collecting the data.
While the static LDA features were shown to be effective at classifying the
datasets, taking both dynamic assembly sequences and high-level behavioral
features would contribute a more complete picture of the programs.
Additionally, high-level behavioral features are more interpretable than
the LDA topics because they convey tangible actions as opposed to the structure
of the underlying code.
The static opcode analysis could be bypassed if the attacker has knowledge of
the detection strategy, but high-level features must occur in some form to
actually perform a given action.

\par Another limitation of the static analysis is that it does not convey any
temporal relationships between events.
In reality, actions will happen in a time sequence over the duration of program
execution.
Having time-series data of high-level behaviors with correlating environmental
information will help to develop a more complex contextual model.

\subsection{Biological Inspiration}%
\label{sec:concl_biological}

\par A huge area to improve on this work is looking at biological inspiration
for the formation of context.
So far, we have relied on statistical methods to implement the context-aware
models.
While we were able to accomplish small pieces of the context problem this way,
it is becoming apparent that statistical methods alone will not be able to
completely solve the problem of context.
Instead, it would be interesting to look for some biological inspiration into
how the brain creates context.
When a reverse engineer looks at a program, they are able to use context they
have developed from past experience as well as knowledge of the overall  system
to determine whether an action is malicious or benign.
The end-goal would be to develop a model which can mimic the way a human expert
can assess the context of a program.
To aid with the biological inspiration, neurologically inspired computational
models should be investigated.

\subsection{Physical Context Model}%
\label{sec:concl_physical_context}

\par One of the most obvious areas for improvement is in the physical context
model, as this work leaves that system as a black box.
Testing should be done to try different kinds of models to see how to most
effectively utilize the data.
In our system, we had the physical context model separated from the rest of the
model, but it may be a better idea to use a model which can directly learn
which actions are allowed in which context.
Exploring neural networks or another model which actually ``learns'' would be
an ideal starting point.

\subsection{Combining Context Models}%
\label{sec:concl_combining_context}

\par At this point, the physical context and expected behavior context is
evaluated in two separate models.
However, it is necessary to consider both of these elements to determine if a
given program is well-behaved given the context.
There are several options, each with varying levels of implementation
difficulty and robustness.
The simplest solution would be to keep the models separate and just say that
a program must pass either individual context check.
However, this solution may not be robust enough, as it considers the context
separately and not wholistically.

\subsection{Parameter Tuning}%
\label{sec:concl_parameter_tuning}

\par In this work, some basic parameter exploration was done on the LDA and
k-NN algorithms.
However, the randomness of LDA fitting caused large error bars for the k-NN
classification accuracy which have significant overlap.
It is possible that further refinement of the LDA parameters aside from just
the number of topics is required to get more consistent results over this type
of data.
Future work should be done to try to refine the parameter tuning.

\newpage

\end{document}
