\documentclass[../stegner_thesis.tex]{subfiles}

\begin{document}

\chapter{Introduction}%
\label{ch:intro}

\section{Motivation}%
\label{sec:intro_motivation}

\par Cyber security is an ever increasing need, especially as society relies
increasingly upon online services.
%dependence on online services is accelerated by the coronavirus pandemic.
In recent times, several notable cyber attacks have demonstrated the need to
further our knowledge of cyber threats.
In the SolarWinds security breach, an adversary was able to spread malicious
software to an estimated 100 companies and several agencies of the U.S.\
federal government~\cite{temple-rastonWorstNightmare}.
It is estimated that this attack could cost the U.S.\ government alone hundreds
of millions of dollars~\cite{nolanEconomicCosts}.
In a separate cyber attack, a major U.S.\ oil pipeline was compromised, leading
to a temporary shutdown of the pipeline~\cite{penalozaCybersecurityAttack2021}.
With such severe consequences, it is vital to increase our ability to detect
and mitigate cyber threats.

\par Detecting whether or not a software is malicious is a difficult task.
The context under which software is executing plays an important role in
understanding malicious behavior.
A given sequence of opcodes or system calls is not necessarily malicious by
nature because malice is relative to the desired outcome of running the
software.
If a specific opcode or system call was malicious, it would simply not be built
into the system in the first place.

\par Take, for example, an autonomous drone with an onboard camera; turning the
camera on and off is not inherently malicious, otherwise the camera would be
hard-wired on.
There can be some times when turning the camera off is a desired, benign
action, such as saving power on the drone or preserving privacy in sensitive
locations.
That being said, a malicious software can cripple the functionality of a drone
by switching off the camera at undesired times.
The determining factor on whether or not turning the camera off is malicious is
the context in which the software chose to turn it off.

\section{Outline}%
\label{sec:intro_outline}

\par \secref{ch:background} outlines the prerequisite background topics,
including malware analysis and the machine learning techniques used in this
work.
Next, a brief literature review of relevant efforts in malware detection and
context-aware software analysis is given.

\par \secref{ch:methods} details our methodology in developing a context-aware
malware detection model.
We discuss the central point of how we should define context, as well as the
rationale for several proof-of-concept context-aware model iterations.

\par \secref{ch:results} presents the results of the models explained in
\secref{ch:methods}, including a discussion of how well context is addressed.

\par Finally, \secref{ch:conclusion} discusses the main takeaways from our work,
including a summary of areas of improvement for future work.

\newpage

\end{document}
